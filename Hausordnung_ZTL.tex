\documentclass[a4paper, 12pt, numbers=withenddot,]{scrartcl}

%% PDF SETUP
\usepackage[pdftex, bookmarks, colorlinks, breaklinks,
pdfusetitle,plainpages=false]{hyperref}
\hypersetup{linkcolor=blue,pdfauthor={Zentrum für Technikkultur Landau},citecolor=blue,filecolor=black,urlcolor=blue,plainpages=false} 
\usepackage[utf8]{inputenc}
\usepackage[T1]{fontenc}
\usepackage[ngerman]{babel}
\usepackage{url}

\usepackage{enumitem}

% Optima as a sans serif font.
\renewcommand*\sfdefault{uop}
\usepackage[protrusion=true,expansion=true]{microtype}
% Recalculate page setup based on new font.
\KOMAoptions{DIV=last}
\pagestyle{plain}

%\renewcommand*\thesection{\arabic{section}}
\KOMAoptions{toc=flat}

\begin{document}
\title{Hausordnung ZTL.space}
\subtitle{(Version 1.0.0)}
\author{}
\date{}

\maketitle

\section{Der ZTL.space und du}
\begin{enumerate}[label=\alph*)]
	\item Der Makerspace und Hackerspace des Zentrums für Technikkultur Landau e. V.
(kurz ZTL.space) ist ein Community-Projekt. Er lebt von dem, was du beiträgst.
Gib dein Wissen weiter und trage so zur Entwicklung des ZTL.space bei.
	\item  Mit dem Betreten des ZTL.space erklärst du dich mit dieser Hausordnung
einverstanden.
	\item Verhalte dich anderen gegenüber so, wie auch du es erwartest behandelt zu
werden.
	\item Wir sind eine Maker-Community, kein Dienstleister. Passe bitte deine
Erwartungshaltung an.
	\item Wir sind dem Gedanken der Open Source Bewegung verbunden. Daher begrüßen
wir es, wenn du deine im ZTL.space entstandenen Projekte der Community unter
freien Lizenzen zur Verfügung stellst.
	\item Wir sehen uns als politisch und religiös neutral und unser ZTL.space ist ein Raum
für Kreativität und Kommunikation. Vorsätzliche politische und religiöse Aktivitäten
sind im ZTL.space nicht erwünscht.
	\item  Wir sind eine demokratische, wissen(schaft)sbasierte und säkulare Organisation.
Antidemokratische, faschistische, sexistische, rassistische oder andere
extremistische Handlungen, Kennzeichen und Reden führen zum Platzverweis.
	\item  Bitte keine Werbung: Du darfst deine Zugehörigkeit zu anderen Organisationen
zeigen, aber unsere Mitglieder möchten weder für deine Religion oder deine
politische Partei angeworben werden.
	\item Laute Musik, Telefonieren, Unterhaltungen, dreckige Tassen, Pizza-Geruch,
Swearing, und was auch immer: Klärt sowas bitte mit-/untereinander. Klappt das
nicht, dann verteilt der Vorstand die Förmchen – und ihr müsst damit leben.
\end{enumerate}

\newpage

\section{Zugang, Aufenthalt und Umgang}
\begin{enumerate}[label=\alph*)]
	\item Der Aufenthalt und das Arbeiten im ZTL.space erfolgt grundsätzlich auf eigene
Gefahr.
	\item Hausrecht hat der Verein Zentrum für Technikkultur Landau e. V. Anwesende, die
sich nicht an die Hausordnung oder an die Anordnungen halten, können der
Räume verwiesen werden. Das Hausrecht wird durch ein Vorstandsmitglied
ausgeübt, ist kein Vorstandsmitglied anwesend wird das Hausrecht durch die
Mehrheit der anwesenden Mitglieder ausgeübt. Der Vorstand kann über ein
permanentes Hausverbot entscheiden.
	\item Der ZTL.space ist geöffnet, solange sich mindestens ein Mitglied in den
Räumlichkeiten befindet. Das letzte anwesende Mitglied muss vor Verlassen der
Räume mit einer angemessenen Vorlaufzeit alle weiteren Anwesenden auffordern,
den ZTL.space zu verlassen.
	\item In allen Räumen des ZTL.space gilt absolutes Rauchverbot.
	\item Körperliche Auseinandersetzungen führen zum sofortigen Hausverbot für alle
Beteiligten, über eine Aufhebung des Hausverbots entscheidet der Vorstand. Die
Beteiligten können sich gegenüber dem Vorstand zu der Angelegenheit äußern.
	\item Die Übernachtung im ZTL.space ist untersagt.
	\item Sollte es im ZTL.space zu Kontakt mit extraterrestrischen Lebensformen kommen,
sind diese gastfreundlich mit Kaffee oder Mate und einem Sitzplatz zu empfangen.
Als Mitglied in den Verein aufgenommen werden können nur Wesen, die einen
vom ZTL erarbeiteten Turing-Test bestehen. In außergewöhnlichen Fällen kann
der Vorstand über Ehrenmitgliedschaften entscheiden. Bei Offenbarung bösartiger
Ziele, speziell im Hinblick auf die Vernichtung der Erde, müssen extraterrestrische
Lebensformen aufgrund des bestehenden Interessenkonfliktes - unter
Berücksichtigung des Punktes „Haftung“ - bekämpft werden. Eine eventuelle
Mitgliedschaft der extraterrestrischen Lebensformen ruht während der
Kampfhandlungen. Da die Existenz außerirdischer Lebensformen als statistisch
gegeben angesehen werden muss, erfüllen Zwischenfälle mit extraterrestrischen
Lebensformen nicht die Kriterien für höhere Gewalt.
\end{enumerate}

\newpage

\section{Sicherheit, Haftung und Funktion}
\begin{enumerate}[label=\alph*)]
	\item Du verpflichtest dich Geräte, die einer Einweisung bedürfen, erst zu benutzen,
nachdem du die Einweisung erhalten hast.
	\item Safety first. Bitte passe auf dich und alle anderen auf. Falls du siehst, dass
Personen ohne angemessene Schutzmaßnahmen, Schutzkleidung oder
Erfahrung an Maschinen oder mit sonstigem Werkzeug arbeiten, dann weise sie
darauf hin. Wirst du auf so etwas hingewiesen, dann sehe das als Wertschätzung
und nicht als Maßregelung. Du darfst dich auch gerne dafür bedanken.
	\item  Verändere bitte nichts an der Infrastruktur des ZTL.space, ohne dass vorher mit
den Verantwortlichen abzusprechen. Verbesserungsvorschläge und die
Bereitschaft mitzuarbeiten werden natürlich dankend angenommen und tragen zur
Verbesserung des ZTL.space bei.
	\item Die Rechner, Systeme und Zugangsbeschränkungen im ZTL.space, einschließlich
derer, die Mitglieder und Gäste mit sich führen, dürfen nicht kompromittiert werden.
	\item  Wenn du Verbrauchsmaterial benutzt, dann fülle es bitte wieder auf oder spende
einen angemessenen Betrag zum Nachkauf und weise die Verantwortlichen auf
das fehlende Material hin.
	\item Es kann – und wird – immer etwas kaputtgehen. Falls dir so etwas passiert, teile
es bitte einer verantwortlichen Person mit. Immer! Wirklich IMMER! Auch wenn
keiner gesehen hat, dass es DIR kaputtgegangen ist. Niemand wird dir einen Strick
daraus drehen, aber es ist wichtig für die Sicherheit aller!
	\item Jedes Mitglied muss alles tun, um Brände zu vermeiden und die Brandgefahr zu
mindern. Geräte dürfen nicht ohne vorherige Absprache mit den Verantwortlichen
dauerhaft betrieben werden. Geräte, die nicht unbeaufsichtigt betrieben werden
sollen, müssen beim Verlassen des ZTL.space ausgeschaltet werden. Installierte
Rauchmelder dürfen nicht manipuliert oder deaktiviert werden. Falls Rauchmelder
durch Tonzeichen leere Batterien ankündigen, muss diese Information den
Verantwortlichen mitgeteilt werden.
	\item Beim Verlassen des ZTL.space muss die Checkliste am Eingang abgearbeitet
werden.
	\item Für Schäden gleich welcher Art, ins besondere wenn sie sich aus Missachtung der
Hausordnung ergeben, haftet der/die Verursacher/in in vollem Umfang. Der
Abschluss einer privaten Haftpflichtversicherung wird empfohlen.
\end{enumerate}

\newpage

\section{Sauberkeit und Ordnung}
\begin{enumerate}[label=\alph*)]
	\item Hinterlasse den ZTL.space und deinen Arbeitsplatz immer sauberer, aufgeräumter
und schöner als du ihn vorfindest. Lege Werkzeug zurück an seinen Platz. Wenn
du Mobiliar verschoben hast, baue das zurück.
	\item Du darfst immer gerne putzen.
	\item Der ZTL.space ist kein Lagerraum. Wenn es nicht anders abgesprochen ist, musst
du deine Sachen nach jedem Besuch wieder mit aus dem ZTL.space nehmen.
	\item Die Küche ist in jedem Fall hygienisch zu hinterlassen. Verdorbene Nahrungsmittel
im Kühlschrank sind zu entsorgen. Nach der Nutzung von Geräten müssen diese
sauber hinterlassen werden. Offene Gebinde müssen mit dem Öffnungsdatum
versehen werden.
	\item Müll muss in die entsprechend gekennzeichneten Behälter sortiert entsorgt
werden.
	\item Wir lieben das freie Internet, Menschen, Tiere, Pflanzen und Steine. Aber aus
Rücksicht auf Allergiker, und im Sinne des Tierwohls, sind Tiere im ZTL.space
nicht erlaubt. Davon ausgenommen sind natürlich Blinden- und Assistenztiere.
\end{enumerate}



\vspace{2.5cm}

\noindent Beschlossen bei der Mitgliederversammlung am 29.08.2020

\end{document}
